\resheading{Course Projects}
\begin{itemize}
\item \textbf{Simulation Analysis of Web Application } \emph{[ CS681 Performance Evaluation of Computer Systems and Networks]} \\
	\emph{(Guided by Prof. Varsha Apte, Spring 2014)} \hfill \\[-0.6cm]
	\begin{itemize}
	  \item Goal : Study the performance of web application through Discrete event Simulation. \\[-0.6cm]
	      \item Technology/Languages/Tools  python, tsung, GNUplot. \\[-0.6cm]
	      \item Description 
	      This project simulated the behavior of traffic on the web server using discrete event simulation model and
	      then its performance compared with real web server by tuning different parameters such as an arrival
	      rate, departure rate, context switch time, queuing delay, timeout, think time, No. of users. \\[-0.6cm]
	\end{itemize}
	
	\item \textbf{Extraction of information about RSS and WSS of a Process in Linux kernel.} \hfill \emph{[ CS401 Kernel Programming ]}\hfill 
	\emph{(Guided by Prof. Purushottam Kulkarni Spring 2015)} \\
	-Implemented loadable kernel module with logic to find out the size of rss and wss of process by using Paging hierarchy. \\[-0.6cm]
	\item \textbf{Analysis of the scheduler of Linux kernel by generating scheduler-level and process-level statistics.} \hfill \emph{[ CS401 Kernel Programming ]} \hfill
	\emph{(Guided by Prof. Purushottam Kulkarni, Spring 2015)}\hfill \\[-0.7cm]
	\begin{itemize}
	 \item Implemented loadable kernel module and hooks to extract information related to scheduler and process.\\[-0.6cm]
	 \item Scheduler level statistics Experiments: Load vs Context Switches, Run Queue length distribution per CPU, Number of Migrations across CPUs, Experiment with scheduling priorities, context switches vs time, number of migration.\\[-0.6cm]
	 \item Process level Statistics : Number of context switches, Variation in dynamic priority of the process, CPU mapping distribution of process, Experiments with CPU affinity of process.\\[-0.6cm]
	\end{itemize}
	\item \textbf{Analysis of TCP Congestion Control.} \hfill \emph{[ CS641 Advanced Computer Network ]} \emph{(Guided by Prof. Mythili V.)} \\[-0.6cm]
\begin{itemize}
	\item Used ns-3 Simulation for collecting data about TCP NewReno and Tahoe. 
\end{itemize}
	\item \textbf{Automation of ns3 simulation and graph generation}\hfill\emph{[ CS699 Software Lab ]}  \\
	\emph{(Guided by Prof. Bhaskaran Raman, Autumn 2014)}  \\[-0.6cm]
	\begin{itemize}
	 \item Project based on : to automate ns3 simulator runs and to make graphs of data which is obtained by parsing of pcap
files. It is related to networking, but the main task is automation of the various components of a simulation .
	 \item Technology used: Bash, Awk,Lex yacc, Python, pyplot,ns3 network simulator, tshark, Makefile, git.
	\end{itemize}
	%\item \textbf{Replication Management In Replicated Distributed Database System}  \hfill \emph{[Btech Final year Project]}\\
	%\hfill \emph{(with Asst. Prof. R.K.Dwivedi)}\hfill \emph{[July 2012 - May 2013 ]} \\[-0.6cm]
	%\begin{itemize}
	%      \item Modified a protocol to make read operation more efficient in terms of time complexity without affecting fault tolerance of the system. \\[-0.6cm]
	%\end{itemize}

		
\end{itemize}
